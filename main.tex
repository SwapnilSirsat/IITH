\documentclass{article}
\usepackage[utf8]{inputenc}
\usepackage{amsmath}

\title{Assignment 1}
\author{Swapnil Sirsat}
\date{January 2021}

\begin{document}
\maketitle
\section*{Question :}

Express the following matrices as the sum of
a symmetric and a skew symmetric matrix:\\
$(i)$ $\begin{pmatrix}
    3 & 5\\
    1 & 1\\
\end{pmatrix}$\\
$(ii)$ $\begin{pmatrix}
    6 & -2 & 2\\
    -2 & 3 & -1\\
    2 & -1 & 3\\
\end{pmatrix}$\\
$(iii)$  $\begin{pmatrix}
3 & 3 & -1 \\
-2 & -2 & 1\\
-4 & -5 & 2\\
\end{pmatrix}$\\
$(iv)$ $\begin{pmatrix}
    1 & 5\\
    -1 & 2\\
\end{pmatrix}$\\
\section*{Solution:}\\ 
\subsection*{$(i)$ $\begin{pmatrix}
    3 & 5\\
    1 & 1\\
\end{pmatrix}$\\}



\\$A = \begin{pmatrix}
    3 & 5\\
    1 & 1\\
\end{pmatrix}$\\ \\
then,  $ A' = \begin{pmatrix}
                3 & 1 \\
                5 & 1 \\
            \end{pmatrix}$\\
let \\
$P$ = \( \frac{1}{2} \)$(A+A')$\\
 $P$ = \( \frac{1}{2} \)( \begin{pmatrix} 3 & 5 \\ 1 & 1 \end{pmatrix} + \begin{pmatrix} 3 & 1 \\ 5 & 1 \end{pmatrix} )\\
 $P$ =  \( \frac{1}{2} \)( \begin{pmatrix} 6 & 6 \\ 6 & 2 \end{pmatrix} )\\
 $P$ =  \begin{pmatrix} 3 & 3 \\ 3 & 1 \end{pmatrix} \\
 now,\hspace{1cm}$P'$ =  \begin{pmatrix} 3 & 3 \\ 3 & 1 \end{pmatrix} = P \\ 
 
 therefore P is symmetric matrix \\
 Now, let$Q$ = \( \frac{1}{2} \)$(A - A')$ \\
 
 $Q$ = \( \frac{1}{2} \)( \begin{pmatrix} 3 & 5 \\ 1 & 1 \end{pmatrix} $-$ \begin{pmatrix} 3 & 1 \\ 5 & 1 \end{pmatrix} )\\ 
 $Q$ = \( \frac{1}{2} \)( \begin{pmatrix} 0 & 4 \\ -4 & 0 \end{pmatrix} )\\
$Q$ =  \begin{pmatrix} 0 & 2 \\ -2 & 0 \end{pmatrix} = $-$ \begin{pmatrix} 0 & -2 \\ 2 & 0 \end{pmatrix} = $Q'$ \\ \\
  therefore $Q$ is a skew-symmetric matrix \\
  now, \\
  $P + Q$ = ( \begin{pmatrix} 3 & 3 \\ 3 & 1 \end{pmatrix} $+$ \begin{pmatrix} 0 & 2 \\ -2 & 0 \end{pmatrix} ) \\ \\ \\
  $P + Q$ = \begin{pmatrix} 
    3 & 5\\
    1 & 1\\
\end{pmatrix} = $A$ \\
thus $A$ is written as $P+Q$ where $P$ and $Q$ are symmetric and skew-symmetric matrices respectively.
 
 
 
 

\newpage
\subsection*{$(ii)$ $\begin{pmatrix}
    6 & -2 & 2\\
    -2 & 3 & -1\\
    2 & -1 & 3\\
\end{pmatrix}$\\}


$ A =\begin{pmatrix}
    6 & -2 & 2\\
    -2 & 3 & -1\\
    2 & -1 & 3\\
\end{pmatrix}$$\\
	
then $A'$ =\begin{pmatrix}
    6 & -2 & 2\\
    -2 & 3 & -1\\
    2 & -1 & 3\\
\end{pmatrix}\\
Now, $A+A' $ = \begin{pmatrix}6 & -2 & 2\\-2 & 3 & -1\\2 & -1 & 3\ \end{pmatrix} $+$ \begin{pmatrix}6 & -2 & 2\\-2 & 3 & -1\\2 & -1 & 3\\ \end{pmatrix} $=$\begin{pmatrix}12 & -4 & 4\\-4 & 6 & -2\\4 & -2 & 6\ \end{pmatrix} \\
let \\
$P$ = $\( \frac{1}{2} \)$(A+A')$ =\( \frac{1}{2} \)$\begin{pmatrix}12 & -4 & 4\\-4 & 6 & -2\\4 & -2 & 6\\ \end{pmatrix}$ $=$ $\begin{pmatrix}6 & -2 & 2\\-2 & 3 & -1\\2 & -1 & 3\ \end{pmatrix}$ \\
Now, $P'$=P
Thus, P is a symmetric matrix

Now $A-A'$ = \begin{pmatrix}6 & -2 & 2\\-2 & 3 & -1\\2 & -1 & 3\ \end{pmatrix} $-$ \begin{pmatrix}6 & -2 & 2\\-2 & 3 & -1\\2 & -1 & 3\\ \end{pmatrix} = \begin{pmatrix}0 & 0 & 0  \\0 & 0 & 0 \\0 & 0 & 0\\ \end{pmatrix}\\

let, $Q$ = \(\frac{1}{2}\)($A-A'$)$= \begin{pmatrix}0 & 0 & 0 \\0 & 0 & 0\\0 & 0 & 0 \\ \end{pmatrix}\\
$Q'$ =  \begin{pmatrix}0 & 0 & 0 \\0 & 0 &0\\0 &0 &0 \\ \end{pmatrix} = -$Q\\
thus, $Q$ is a skew symmetric matrix\\
Representing A as sum of P and Q:\\
$A$ = $P+Q$ =  \begin{pmatrix}6 & -2 & 2\\-2 & 3 & -1\\2 & -1 & 3\ \end{pmatrix}+  \begin{pmatrix}0 & 0 &0 \\0 & 0 & 0 \\0 & 0 & 0 \\ \end{pmatrix} =  \begin{pmatrix}6 & -2 & 2\\-2 & 3 & -1\\2 & -1 & 3\ \end{pmatrix} = A\\
\vspace{6cm}

\newline

\subsection*{$(iii)$  $\begin{pmatrix}
3 & 3 & -1 \\
-2 & -2 & 1\\
-4 & -5 & 2\\
\end{pmatrix}$\\}


$A$ = $\begin{pmatrix}3 & 3 & -1 \\-2 & -2 & 1\\-4 & -5 & 2\\\end{pmatrix}$\\  

$A'$ = 	\begin{pmatrix}

  
3&
-2&
-4 \\
  
3&
-2&
-5 \\
	
  
-1&
1&
2\\
	
  

\end{pmatrix} \\


Now, $A+A'$ = $\begin{pmatrix}3 & 3 & -1 \\-2 & -2 & 1\\-4 & -5 & 2\\\end{pmatrix}$ $+$ $\begin{pmatrix}3 & -2 & -4 \\3 & -2 & -5\\-1 & 1 & 2\\\end{pmatrix}$\\ 

$A+A'$ = \begin{pmatrix} 
6&
1 &
-5\\
	
  
1&
−4&
−4\\

  
−5&
−4&
4\\
 \end{pmatrix} 


$P$ = \( \frac{1}{2} \)  $(A + A' )$ \\
$P = $\( \frac{1}{2} \)  ( \begin{pmatrix}
6& 1& -5\\
1& -4& -4\\
-5& -4& 4\\
\end{pmatrix} ) $
\\
P = \begin{pmatrix}
3 & \frac{1}{2} & -\frac{5}{2}\\
\frac{1}{2} & -2 & -2 \\
-\frac{5}{2} & -2 & 2 \\
\end{pmatrix} \\
Now, $P'$ = \begin{pmatrix}
3 & \frac{1}{2} & -\frac{5}{2}\\
\frac{1}{2} & -2 & -2 \\
-\frac{5}{2} & -2 & 2 \\
\end{pmatrix} = $P$\\
Thus P is Symmetric matrix

Now $(A - A')$ = \begin{pmatrix}
0 & 5& 3 \\
-5 & 0 & 0 \\
-3 & -6 & 0 \\
\end{pmatrix}
\\
therefore, $Q$ = $ \( \frac{1}{2} \)  $(A - A' )$ \\
$Q$ =  \( \frac{1}{2} \)\begin{pmatrix}
0 & 5& 3 \\
-5 & 0 & 0 \\
-3 & -6 & 0 \\
\end{pmatrix}
$Q$ = \begin{pmatrix}
0 & \frac{5}{2}& \frac{3}{2} \\
-\frac{5}{2} & 0 & 0 \\
-\frac{3}{2} & -3 & 0 \\
\end{pmatrix}

$Q'$ = \begin{pmatrix}
0 & -\frac{5}{2}& -\frac{3}{2} \\
\frac{5}{2} & 0 & 0 \\
\frac{3}{2} & 3 & 0 \\
\end{pmatrix} = $-Q$\\
therefore Q is skew-Symmetric Matrix
\\Now,\\
$P + Q$ = \begin{pmatrix}
3 & \frac{1}{2} & -\frac{5}{2}\\
\frac{1}{2} & -2 & -2 \\
-\frac{5}{2} & -2 & 2 \\
\end{pmatrix} $+$  \begin{pmatrix}
0 & \frac{5}{2}& \frac{3}{2} \\
-\frac{5}{2} & 0 & 0 \\
-\frac{3}{2} & -3 & 0 \\
\end{pmatrix} $=$ \begin{pmatrix}3 & 3 & -1 \\-2 & -2 & 1\\-4 & -5 & 2\\\end{pmatrix} = A \\
thus $A$ is written as $P+Q$ where $P$ and $Q$ are symmetric and skew-symmetric matrices respectively.



\newpage
\subsection*{$(iv)$ $\begin{pmatrix}
    1 & 5\\
    -1 & 2\\
\end{pmatrix}$\\}



\\$A = \begin{pmatrix}
    1 & 5\\
    -1 & 2\\
\end{pmatrix}$\\ \\
then,  $ A' = \begin{pmatrix}
                1 & -1 \\
                5 & 2 \\
            \end{pmatrix}$

let \\
$P$ = \( \frac{1}{2} \)$(A+A')$\\
 $P$ = \( \frac{1}{2} \)( \begin{pmatrix} 1 & 5 \\ -1 & 2 \end{pmatrix} + \begin{pmatrix} 1 & -1 \\ 5 & 2 \end{pmatrix} )\\
 $P$ =  \( \frac{1}{2} \)( \begin{pmatrix} 2 & 4 \\ 4 & 4 \end{pmatrix} )\\
 $P$ =  \begin{pmatrix} 1 & 2 \\ 2 & 2 \end{pmatrix} \\
 now,\hspace{1cm}$P'$ =  \begin{pmatrix} 1 & 2 \\ 2 & 2 \end{pmatrix} = P \\ 
 
 therefore P is symmetric matrix \\
 Now, let$Q$ = \( \frac{1}{2} \)$(A - A')$ \\
 
 $Q$ = \( \frac{1}{2} \)( \begin{pmatrix} 1 & 5 \\ -1 & 2 \end{pmatrix} $-$ \begin{pmatrix} 1 & -1 \\ 5 & 2 \end{pmatrix} )\\ 
 $Q$ = \( \frac{1}{2} \)( \begin{pmatrix} 0 & 6 \\ -6 & 0 \end{pmatrix} )\\
$Q$ =  \begin{pmatrix} 0 & 3 \\ -3 & 0 \end{pmatrix} = $Q'$\\ 
  therefore $Q$ is a skew-symmetric matrix \\
  now, \\
  $P + Q$ = ( \begin{pmatrix} 1 & 2 \\ 2 & 2 \end{pmatrix} $+$ \begin{pmatrix} 0 & 3 \\ -3 & 0 \end{pmatrix} ) \\ \\ \\
  $P + Q$ = \begin{pmatrix}
    1 & 5\\
    -1 & 2\\
\end{pmatrix} = $A$ \\
 thus $A$ is written as $P+Q$ where $P$ and $Q$ are symmetric and skew-symmetric matrices respectively.
 
 
 


\end{document}


\newpage

